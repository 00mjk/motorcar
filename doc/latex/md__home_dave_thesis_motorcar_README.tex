Motorcar is a framework for 3\-D windowing built on top of Wayland which I originally developed for my Master's thesis at Cal Poly (\href{https://github.com/evil0sheep/MastersThesis/blob/master/thesis.pdf?raw=true}{\tt pdf}, \href{https://docs.google.com/presentation/d/1svgGMxxbfmcHy_KuS5Q9hah8PQOsXqvjBKOoMIzW24Y/edit?usp=sharing}{\tt defense slides}). It is designed to provide basic 3\-D windowing infrastructure that gives 3\-D applications desktop flexibility in the how their 3\-D content is drawn while supporting unmodified Wayland applications in the same 3\-D compositor space, and to do this with the simplest mechanism possible.

Motorcar is free and open source (under the B\-S\-D license), and I am very open to contributions (both conceptual and functional) from the community. If you have any questions, comments, or critical feedback about the software, or if you are interested in using or contributing Motorcar, or if you are working on something related, please feel free to contact me, I would love to hear from you.

\section*{Building Motorcar }

In general if you are looking to build Motorcar you should pull the stable branch, as the master branch may periodically go into unbuildable states. Please note that the software in the stable branch is not guaranteed to {\itshape actually be} stable, but I will not knowingly commit broken code to stable.

If you have trouble building Motorcar or feedback regarding the build process please contact me. At the time of me writing these build instructions this software has not been built on another system to my knowledge, so there could be significant problems with the software or build system that I am simply unaware of.

\subsection*{Dependencies }

Motorcar has significant external dependencies, some of which may need to be built from source depending on your Linux distribution. These dependencies can be summarized at a high level as a dependency on the \href{http://qt-project.org/wiki/QtWayland}{\tt Qt\-Wayland Qt\-Compositor} module with support for E\-G\-L with desktop Open\-G\-L (not Open\-G\-L E\-S). This in turn requires \href{http://wayland.freedesktop.org/}{\tt Wayland}, which may be available as a package depending on your Linux distribution, or may need to be \href{http://wayland.freedesktop.org/building.html}{\tt built from source}, as well as support for E\-G\-L with desktop Open\-G\-L, which may require building Mesa from source. Instructions for building Mesa for Wayland E\-G\-L can be found in the \href{http://wayland.freedesktop.org/building.html}{\tt Wayland build instructions}. I use the following configurations\-:


\begin{DoxyItemize}
\item Mesa \begin{DoxyVerb}  $ autogen.sh --prefix=$WLD --enable-gles2 --disable-gallium-egl --with-egl-platforms=x11,wayland,drm --enable-gbm --enable-shared-glapi --with-gallium-drivers=r300,r600,swrast,nouveau --enable-glx-tls
\end{DoxyVerb}

\item Cairo \begin{DoxyVerb}  $ autogen.sh --prefix=$WLD --enable-xcb=yes --enable-gl=yes -enable-egl=yes
\end{DoxyVerb}

\end{DoxyItemize}

Build instructions for Qt\-Wayland can be found on the \href{http://qt-project.org/wiki/QtWayland}{\tt Qt\-Wayland page}, and these cover most everything needed to get build Qt\-Wayland. Getting it to work properly with E\-G\-L and desktop Open\-G\-L is a little bit tricky, and I have copied my build command below for reference, which I imagine will probably work on most systems, though this is of course not guaranteed.

Below are the hashes of the commits which I am currently using for the Qt dependencies, which I include mainly for reference since they are known to work together. Other combinations may work as well. Please note that qtbase and qtwayland are git submodules of the qt5 repository.


\begin{DoxyItemize}
\item Qt5\-:
\begin{DoxyItemize}
\item e198c124d3259dea657fcfa4c9b9b43bcd2d9fd0
\item The most recent commit might work here but if the version exceeds 5.\-3.\-0 you may run into problems with the Qt\-Wayland private includes
\end{DoxyItemize}
\item qtbase
\begin{DoxyItemize}
\item 625002f7067271b8f03f7bfa13baff6128c72e68
\item The most recent commit will probably work here
\end{DoxyItemize}
\item qtwayland
\begin{DoxyItemize}
\item 5c605d363e2fc42f5ec80413d093b69614027da4
\item This is the most likely to cause problems as the Compositor A\-P\-I changes quite frequently.
\end{DoxyItemize}
\end{DoxyItemize}

Here is the build sequence I use for Qt5 and Qt\-Wayland to get them to support E\-G\-L with desktop Open\-G\-L. Again, this is certainly not guaranteed to work on every system, it is included here mainly as a reference. Please refer to the \href{http://qt-project.org/wiki/Building_Qt_5_from_Git}{\tt Qt5 build instructions} and the \href{http://qt-project.org/wiki/QtWayland}{\tt Qt\-Wayland build instructions} for more information. \begin{DoxyVerb}$ git clone git://gitorious.org/qt/qt5.git qt5
$ cd qt5
$ git checkout e198c124d3259dea657fcfa4c9b9b43bcd2d9fd0
$ ./init-repository --no-webkit  --module-subset=qtbase,qtjsbackend,qtdeclarative,qtwayland
\end{DoxyVerb}


Other modules may be required to run some of the Qt clients, but not for Motorcar itself \begin{DoxyVerb}$ cd qtbase
$ git checkout 625002f7067271b8f03f7bfa13baff6128c72e68
$ cd ../qtwayland
$ git checkout 5c605d363e2fc42f5ec80413d093b69614027da4
$ cd ../
$ ./configure -prefix /opt/qt5  -debug -confirm-license -opensource -egl -opengl  -no-xcb-xlib
\end{DoxyVerb}


I install into opt/qt5, but this is not a hard requirement, just make sure that when you run qmake it is running the executable installed here. The -\/no-\/xcb-\/xlib argument is required to build against E\-G\-L and desktop Open\-G\-L in the commit listed above, but this may have been fixed in newer versions of Qt\-Wayland \begin{DoxyVerb}$ cd qtwayland
$ git clean -fdx
$ ../qtbase/bin/qmake CONFIG+=wayland-compositor
$ cd ../
$ make
$ make install
\end{DoxyVerb}


If this gives you trouble you might want to try making and installing Qt\-Wayland from the qtwayland subdirectory.

\subsection*{Building Motorcar Itself }

Motorcar is separated into several components, which are designed to be able to be used together or independently from one another. The Wayland protocol extensions used for 3\-D windowing and 3\-D input are specified in \href{https://github.com/evil0sheep/motorcar/blob/stable/src/protocol/motorcar.xml}{\tt motorcar.\-xml} and the language bingings used by the compositor and clients are generated when those components are compiled.

\subsubsection*{Building the Motorcar Compositor Library}

The compositor is built in two steps. The first step builds the Motorcar compositor library which contains the Wayland backend and Qt dependency, and most of the scenegraph and compositing logic. The second step builds the compositor itself, which is a lightweight program that essentially just uses the compositor library to set up the scene and insert devices into the scenegraph.

This allows many compositors to be built with the core windowing infrastructure, and allows developers implementing compositors to add support for their devices or to replace components of the windowing logic (like the window manager) with their own classes without those classes needing to be in the core Motorcar code base (though I am very open to pull requests). It also keeps device specific dependencies out of the core compositor code.

Currently the entire compositor library is built with qmake, but eventually I would like to transition to G\-N\-U autotools and only use qmake to build the Qt dependent components (since this is a relatively small portion of the code base). To build the motorcar compositor library\-: \begin{DoxyVerb}$ cd path/to/motorcar/repo
$ qmake
$ make
\end{DoxyVerb}


This will build shared objects in the lib directory (under the repository root directory) which compositors will link against when using the Motorcar compositor library.

\subsubsection*{Building the Example Compositor}

This repository contains an example compositor which uses an Oculus Rift as a 3\-D display and a Sixense/\-Razer Hydra as a 6\-Do\-F input device. Building this example compositor requires that both the Oculus\-V\-R S\-D\-K and the Sixense S\-D\-K be present in the system. To build the example compositor\-: \begin{DoxyVerb}$ cd path/to/motorcar/repo
$ cd src/examples/compositors/rift-hydra-compositor
$ make LIBOVRPATH=/path/to/LibOVR SIXENSEPATH=/path/to/sixenseSDK_linux_OSX
\end{DoxyVerb}


This will build an executable called rift-\/hydra-\/compositor in the current directory

In the future there will likely be more compositors, and possibly some mechanism for automatically detecting which device S\-D\-Ks are present and auto generating a compositor which reads in some kind of scene configuration file to control scene layout.

\subsubsection*{Building the Motorcar Demo Client}

This repository also contains an example client which supports the Motorcar protocol extensions for view-\/dependent depth-\/composited 3\-D rendering based on weston-\/simple-\/egl. Eventually this client will be broken apart into a client-\/side windowing library and example clients built with this library, similar to the compositor code, and when this happens these build instructions will be updated accordingly. To build the demo client \begin{DoxyVerb}$ cd path/to/motorcar/repo
$ cd src/examples/clients/simple-egl/
$ make 
\end{DoxyVerb}


This will build an executable called motorcar-\/demo-\/client in the current directory. This client takes several flags, most of which were inherited from weston-\/simple-\/egl. The -\/p flag enables portal clipping mode, rather than the default cuboid clipping mode, and the -\/d flag disables depth compositing on the 3\-D window. See \href{https://github.com/evil0sheep/MastersThesis/blob/master/thesis.pdf?raw=true}{\tt my thesis} and my \href{https://docs.google.com/presentation/d/1svgGMxxbfmcHy_KuS5Q9hah8PQOsXqvjBKOoMIzW24Y/edit?usp=sharing}{\tt thesis defense slides} for a conceptual explanation of what this means. 